% Introduction

\chapter{Introduction} % Main chapter title

\section{Overview}
The World Wide Web has been central to the development of the Information Age and is the primary tool billions of people use to interact on the Internet. The huge growth of Web in the last years has lead to the demand for specialized applications, which operate in different devices, are secure and guarantee the constant provision of services to the user, under any circumstances. The software development 
process can be delivered at the same time by multiple stakeholders, which are located in different areas of the world, studying and working on the same code. Meanwhile, the time consuming stages of production necessitate the proper software design, with the purpose of minimizing the possibility of large design flaws, or discovering them in early stages. \par 
	The model driven software development is a design technique which is based in model utilization and assists in the creation of software, which is extensible, reusable and comprehensible by different stakeholders. It contributes in the development of isolated modules with seperated responsibilities. Hence, this design approach is an ideal solution for creating carefully developed, well tested software, which can be upgraded at any moment. \par 
	Another challenge of the time we are in, is the daily touch with large amounts of complex data, which one must constantly study in order to draw conclusions and make decisions. Because of the way the human brain processes information, using charts or graphs to visualize data is easier than poring over spreadsheets or reports. Data visualization is the presentation of data in a pictorial or graphical format. It enables decision makers to see analytics presented visually, so they can grasp difficult concepts or identify new patterns. \par
	In this thesis, we developed a data visualization framework, which follows the principles of model driven programming. The hierarchical data format is utilized for storing complex multidimensional datasets. The framework implements a method to present visually the datasets to the users, which enables easy understanding, conclusion extraction and decision making. \par 
	The framework consists of modules which contain generic code and are configurable by other modules. The framework's functionality is expanded, either through the development of a new independent module, or via the growth of an existing one. The model driven approach enables the system's adaptabibity based in the designer's requirements. The framework presentation is implemented through the creation of a web application, which operates as a data visualization tool, for networks of users working in the same projects. The framework, as well as the application, are mainly developed in the javascript programming language.


%I terastia anaptuksi tou exei odigisei stin anagki gia dimiourgia ekseidikeumenwn efarmogwn, oi opoies leitourgoun se polles diaforetikes suskeues, einai asfaleis kai egguontai tin sunexi paroxi upiresiwn sto xristi upo opoiesdipote sunthikes. I idia i anaptuksi efarmogwn pleon mporei na ginetai apo apo pollous stakeholders tautoxrona, oi opoioi vriskontai se diaforetika simeia tou kosmou kai meletoun k epemvainoun se koino kwdika. Parallila, ta xronovora stadia paragwgis kathistoun megali proteraiotita ton swsto sxediasmo tou logismikou, etsi wste na elaxistopoiithoun i na anakalufthoun sta prwta stadia, megala sxediastika flaws. \par
%	To model-driven software development einai mia sxediastiki texniki i opoia vasizetai stin anaptuksi montelwn, kai voithaei stin paragwgi logismikou to opoio einai epektasimo, reusable kai eukola katanoito apo diaforetikous stakeholders. Suneisferei stin dimiourgia modules kwdika isolated metaksu tous kai me ksexwrista responsibilities. Epomenws, to sugekrimeno sxediastiko approach einai idaniki lusi gia tin dimiourgia prosektika grammenou tested logismikou pou mporei ana pasa stigmi na anavathmistei. \par
%	Ena allo xaraktiristiko tis epoxis pou dianioume einai i kathimerini epafi me large amounts of complex data, ta opoia prepei kaneis na meletaei wste na vgazei sumperasmata kai na pairnei apofaseis. Because of the way the human brain processes information, using charts or graphs to visualize data is easier than poring over spreadsheets or reports. Data visualization is the presentation of data in a pictorial or graphical format. It enables decision makers to see analytics presented visually, so they can grasp difficult concepts or identify new patterns. \par
%
%	In this thesis, anaptusoume ena data visualization framework, to opoio akolouthei tis arxes tou montelostrafous programmatismou. Xrisimopoiei to hierachical data format, gia na apothikeusei complex multidimentional datasets. To framework ulopoiei mia methodo dataset visualization stous xristes, to opoio epitrepei tin eukoli katanoisi, eksagwgi sumperasmatwn kai decision making. \par 
%	Gia tin anaptuksi tou, dimiourgithikan modules ta opoia periexoun generic kwdika kai einai parametropoiisima apo alla modules. Auto exei ws apotelesma tin paragwgi reusable kai extensible modules ta opoia sundiazontai metaksu tous gia tin dimiourgia tou framework. I parousiasi tou framework ginetai mesw tis dimiourgias mias web efarmogis, i opoia leitourgei ws ena data visualization tool gia diktua apo users oi opoioi ergazontai se koina projects. To framework, opws kai i efarmogi, ulopoiithikan sto megalutero vathmo se javascript.
\section{Outline}
%Chapter ~\ref{chapter2_bg} provides the necessary theoretical background used throughout this thesis, including model driven software development, data access object, access control list and RESTful web service. An overview of the related work and technologies used is presented in chapter ~\ref{chapter3_rw}. The proposed detailed framework architecture is described in chapter ~\ref{chapter4_d}. In chapter~\ref{chapter5_i} we discuss the framework and application implementation as well as the technologies we utilized. Next, we evaluate the framework performance in chapter~\ref{chapter6_pe}. Finally, the conclusion and future work are presented in chapter~\ref{chapter7_cf}.