\chapter{Conclusions and Future Work}
\label{chapter7_cf}
\section{Conclusions}
In the scope of this diploma thesis we designed and implemented a back and front end framework, globally developed in javascript, using the model driven software development approach. The framework provides entity to model mapping features, RESTful web service, data access object and CRUD operations for each model. The system utilizes a custom access control list module for user authorization. Also, the framework implements a method, which represents multidimensional HDF files in plots, in an efficient and integrated way. The front end tier of the framework utilizes the single page application design approach. \par 
	In order to present the framework, not only as an idea but also as an implemented concept, we designed and developed a demo application. The application's objective is the creation of user networks, in order to manipulate and visualize multidimensional datasets. Following we present the main advantages of our framework.

\paragraph{Extensibility} The framework follows the principles of model driven software development architecture, so that any extra operation can be added, either within the context of an already developed service or as a new functionality, with a few lines of code. For example, an extra entity model or a route method may be easily added into the system.
	

\paragraph{No context switching} The framework is developed mainly in javascript. Since only one programming language must be learned in order to evolve the current framework, the developer's productivity is increased. Especially the front end tier was developed exclusively in pure javascript from scratch, without the usage of a large framework.


\section{Future Work}
The way in which the framework is designed and implemented, offers us the potentiality of functionality extension in the future. Below we mention some possible extensions.

\paragraph{Server clustering}  A single instance of Node.js runs in a single thread. To take advantage of multi-core systems, the user will sometimes want to launch a cluster of Node.js processes to handle the load. The node.js cluster module can be used to utilize this functionality, so that the system scales on full load.

\paragraph{Mongodb replication and sharding} Sharding complications and replication fault tolerance will be studied in the future to optimize scalability.

\paragraph{Alternative methods in existing functionalities}
The framework functionality can be expanded in current operations. Upload file formats may be added (i.e csv files), in order to be converted and saved in the framework filesystem. Also, we can use different methods in plot sampling, such as mean values.
 
\paragraph{Image extraction for usage outside the system}
Right now the interaction between system users and no users is not possible. In the future, extra features, such as exporting plots in jpg format to share with users outside the application, may be added to solve this problem.

	